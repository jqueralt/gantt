\documentclass[a4paper,%
	      10pt,%
              %DIV=14,%
              parskip=half,%
]{scrartcl}
\usepackage[utf8]{inputenc}
\usepackage[catalan]{babel}
%\usepackage{amsmath,amsfonts,amsthm,amssymb}

%%%%%%%%%%%%%%%%%%%%%%%%%%%%%%%%%%%%%%%%%%%%%%%%%%%%%%%%%
\usepackage{tgschola}
%\usepackage[bitstream-charter]{mathdesign}
\usepackage{tgheros}
\usepackage{tgcursor}
%Century Schoolbook L
%%% colors
\usepackage[dvipsnames]{color}

\usepackage[pdftex,%                           % paquet per fer aparèixer els colors
		dvipdfm,%
		dvipsnames,%
		svgnames,%
		usenames,%
		table%
		]{xcolor}

%\definecolor{linkcolor}{rgb}{0,0,0.42}        %color de l'enllaç
\colorlet{linkcolor}{RoyalBlue!50!black}
%%%%%%%%%%%%%%%%%%%%%%%%%%%%%%%%%%%%%%%%%%%%%%%%%%%%%%%%%%%%%%%%%%%%%%%%%
\usepackage{pgfgantt}
% ftp://ftp.di.uminho.pt/pub/ctan/graphics/pgf/contrib/pgfgantt/pgfgantt.pdf
%%%%%%%%%%%%%%%%%%%%%%%%%%%%%%%%%%%%%%%%%%%%%%%%%%%%%%%%%%%%%%%%%%%%%%%%%

\usepackage{hyperref}    

\pdfstringdef{\Titol}{Cronogrames o Diagrames de Gantt}
\pdfstringdef{\Tema}{Diagrames de Gantt pel seguiment de projectes}%
\pdfstringdef{\Claus}{Gantt, diagrames, cronogrames}%

\hypersetup{%
    pdftitle=\Titol,%
    pdfsubject=\Tema,%
    pdfkeywords=\Claus,%
    pdfauthor={\copyright 2012 Joan Queralt},% 
    bookmarksopen=true, %finestreta de marcadors oberta quan s'obre el document
   % backref=true,
    linktocpage=true,
    urlcolor=linkcolor,
    citecolor=linkcolor,
    linkcolor=linkcolor,
    colorlinks=true,
  }
%opening
\title{\Titol}
\author{Joan Queralt Gil\\cata\LaTeX{}}
\date{\today}

\begin{document}

\maketitle

\section{Presentació}
Diu la \href{http://ca.wikipedia.org/wiki/Diagrama_de_Gantt}{Vikipèdia} que \textit{el diagrama de Gantt és una de les tècniques usades tant per l'administració pública com per l'empresa privada com a eina de gestió de la qualitat. Concretament, és una eina de planificació del treball, ja que presenta totes les activitats que s'han de realitzar i quan s'han de realitzar, i permet tenir una idea de com avança el projecte i si és necessari reprogramar les actuacions planificades per tal d'adequar el projecte al nou entorn o necessitats. S'ha adaptat al treball per projectes en educació, repartint les feines del grup}.

El paquet \verb+pgfgantt+ proporciona l'entorn \verb+ganttchart+, que
dibuixa un diagrama de Gantt en una imatge TikZ. 

L'usuari pot afegir diversos elements a la taula, com ara:
\begin{itemize}
 \item títols (\verb+\gantttitle+)
 \item barres (\verb+\ganttbar+)
 \item fites o objectius (\verb+\ganttmilestone+)
 \item grups (\verb+\ganttgroup+)
 \item enllaços entre aquests elements (\verb+\ganttlink+)
\end{itemize}

A més, els elements del gràfic són altament personalitzables a causa d'una
gran varietat d'opcions.


\section{Exemple 1}
\begin{ganttchart}%
%opcions del gràfic
[y unit title=0.4cm,                        % alçada de la caixa del títol
y unit chart=0.5cm,                       % alçada de les divisions (=mesos)
vgrid,                                             % mostra graella vertical
hgrid,                                             % mostra graella horitzontal
title/.style={draw=none, fill=RoyalBlue!50!black}, % estil del títol
title label font=\sffamily\bfseries\color{white},  % format del títol
title label anchor/.style={below=-1.6ex},
title left shift=.05,
title right shift=-.05,
title height=1%
]{16}%                                             % nombre de divisions verticals (4+12)
% Títols
\gantttitle{2012}{4}                               % títol i nombre de divisions que ocupa=4 
\gantttitle{2013}{12} \\                           % títol i nombre de divisions que ocupa=12   
\gantttitlelist{9,...,12,1,2,3,4,5,6,7,8,9,10,11,12}{1} \\ % títols de les divisions (=mesos)
% grups:
\ganttgroup[group/.style={draw=black, fill=OliveGreen}]%     estil del grup
           {Fase Pilot}{1}{4}\\ %      Títol del grup 1 amb divisió inici=1 i final=4
\ganttgroup[group/.style={draw=black, fill=OliveGreen!60}]%  estil del grup
           {Fase Producció}{5}{16}\\ % Títol del grup 2 amb divisió inici=5 i final=16
% objectiu: 
\ganttmilestone[name=obj1,% nom de l'objectiu
               milestone/.style={fill=red, draw=black, rounded corners=3pt}]% estil
               {Publicació al web}{4} \\ % Títol de l'objectiu i divisió on situar-lo
% barra:
\ganttbar[name=b1,% nom de la barra
         bar/.style={fill=Purple}% estil
         ]{Comité de pilotatge}{1}{4} \\% Títol de la barra amb divisió d'inici=1 i final=4
% objectiu:
\ganttmilestone[name=obj2]{Criteris de creació}{4}\\% Nom de l'objectiu, Títol i divisió on situar-lo
% objectiu:
\ganttmilestone[name=obj3]{Producte pilot}{4}\\% Nom de l'objectiu, Títol i divisió on situar-lo                     
% enllaç
\ganttlink{b1}{obj3}\\% Enllaça la barra b1 i l'objectiu obj3 
% barra 2:
\ganttbar[name=b2,% 
         bar/.style={fill=Purple!60!white}%
         ]{Comité de seguiment}{5}{16} \\
% barra 3:
\ganttbar[name=b3,% 
         bar/.style={fill=blue}%
         ]{Instal·lació línia producció}{5}{7} \\
% objectiu 4:
\ganttmilestone[name=obj4]{Posada a la venda}{7}\\  
\ganttlink{b3}{obj4}\\
% objectiu 5:
\ganttmilestone[name=obj5,milestone/.style={fill=red, draw=black, rounded corners=3pt}]%
               {Venda a tots els punts de la ciutat}{8}\\ %objectiu 5
\ganttlink{obj4}{obj5}\\
\ganttbar[name=b4,% 
         bar/.style={fill=green!70!black}%
         ]{Elaboració del producte}{8}{10} \\
% objectiu  6
\ganttmilestone[name=obj6]{Venda a tots els punts del país}{10}\\     
% enllaç               
\ganttlink{b4}{obj6}\\  
% objectiu 7
\ganttmilestone[name=obj7,milestone/.style={fill=red, draw=black, rounded corners=3pt}]{Promoció als punts de venda}{10}\\
% enllaç
\ganttlink{obj6}{obj7}\\
% barra 5
\ganttbar[name=b5,% 
               bar/.style={fill=brown}%
               ]{Estudi d'acceptació del mercat}{13}{16} \\
% objectiu 8
\ganttmilestone[name=obj8]{Anàlisi dels resultats}{16}\\ 
\ganttlink{b5}{obj8}\\ 
 %objectiu 9                   
\ganttmilestone[name=obj9]{Elaboració de noves propostes}{16}\\   
% enllaç                
\ganttlink{obj8}{obj9}\\
\end{ganttchart}

%%%%%%%%%%%%%%%%%%%%%%%
\section{Exemple 2}
%%%%%%%%%%%%%%%%%%%%%%%
\begin{ganttchart}%
%opcions del gràfic
[y unit title=0.8cm,                               % alçada de la caixa del títol
y unit chart=0.5cm,                                % alçada de les divisions (=mesos)
x unit =1cm,				          % amplada divisions horitzontals
vgrid,                                             % mostra graella vertical
%hgrid,                                             % mostra graella horitzontal
title/.style={draw=none, fill=Red!50!black}, % estil del títol
title label font=\sffamily\bfseries\color{white},  % format del títol
title label anchor/.style={below=-1.6ex},
title left shift=.05,
title right shift=-.05,
title height=1%
]{7}%                                             % nombre de divisions verticals (7 dies)
% Títols
\gantttitle{Setmana 23}{7} \\           % títol i nombre de divisions que ocupa=7  
\gantttitlelist{9,...,15}{1} \\         % títols de les divisions (=dies)
% grups:
\ganttgroup[group/.style={draw=black, fill=OliveGreen}]%     estil del grup
           {Laborables}{1}{5}\\ %      Títol del grup 1 amb divisió inici=1 i final=5
\ganttgroup[group/.style={draw=black, fill=red}]%  estil del grup
           {Cap de setmana}{6}{7}\\ % Títol del grup 2 amb divisió inici=5 i final=16
\\% línia en blanc
% barra:
\ganttbar[name=b1,% nom de la barra
         bar/.style={fill=Purple}% estil
         ]{Posada en marxa}{1}{3}\\% Títol de la barra amb divisió d'inici=1 i final=4
% objectiu: 
\ganttmilestone[name=obj1,% nom de l'objectiu
               milestone/.style={fill=red, draw=black, rounded corners=3pt}]% estil
               {Reunió inicial}{0} \\ % Títol de l'objectiu i divisió on situar-lo
% objectiu:
\ganttmilestone[name=obj2]{Prova pilot}{1}\\% Nom de l'objectiu, Títol i divisió on situar-lo
% objectiu:
\ganttmilestone[name=obj3]{Producte pilot}{2}\\% Nom de l'objectiu, Títol i divisió on situar-lo                     
\ganttbar[name=b2,% nom de la barra
         bar/.style={fill=Purple}% estil
         ]{Avaluació}{4}{7}\\ 
% objectiu:
\ganttmilestone[name=obj4]{Preparació línia de produciió}{3}\\
% objectiu:
\ganttmilestone[name=obj5]{Posada a la venda}{6}\\
% objectiu:
\ganttmilestone[name=obj6]{Estudi de mercat}{7}\\
% enllaç
\ganttlink{obj1}{obj2}
% enllaç
\ganttlink{obj2}{obj3}
% enllaç
\ganttlink{obj3}{obj4}
% enllaç
\ganttlink{obj4}{obj5} 
% enllaç
\ganttlink{obj5}{obj6} 
% enllaç
\ganttlink{b1}{b2}
\end{ganttchart}

\section{Exemple 3}

\definecolor{barblue}{RGB}{153,204,254}
\definecolor{groupblue}{RGB}{51,102,254}
\definecolor{linkred}{RGB}{165,0,33}
\renewcommand\sfdefault{phv}
\renewcommand\mddefault{mc}
\renewcommand\bfdefault{bc}
\sffamily
% traduccions per no tocar el fitxer sty:
\ganttset{progress label text={#1\% completat}}
\setganttlinklabel{s-s}{començar i començar}
\setganttlinklabel{f-s}{acabar i començar}
\setganttlinklabel{f-f}{acabar i acabar}
\begin{ganttchart}%
[%
% estil del marc
canvas/.style={%
fill=none,%
draw=black!5,%
line width=.75pt},%
% graella horitzontal
hgrid style/.style={draw=black!5, line width=.75pt},%
% graella vertical
vgrid={*1{draw=black!5, line width=.75pt}},%
% marca vertical de la data d'avui
today=9.1,
% estil de la data d'avui
today rule/.style={draw=Blue!50!black,
dash pattern=on 3.5pt off 4.5pt, 
line width=2.5pt},
% etiqueta per la data d'avui
today label={\small\bfseries AVUI},
% TÍTOLS
% estil del títol
title/.style={draw=none, fill=none},
% tipus de lletra
title label font=\bfseries\footnotesize,
% posició
title label anchor/.style={below=7pt},
include title in canvas=false,
% BARRES
bar label font=\mdseries\small\color{black!70},
bar label anchor/.style={left=2cm},
bar/.style={%draw=none, 
fill=black!63},
bar incomplete/.style={fill=barblue},
progress label font=\mdseries\footnotesize\color{black!70},
% GRUPS
group incomplete/.style={draw=black,fill=groupblue},
group left shift=0,
group right shift=0,
group height=.5,
group peaks={0}{}{},
group label anchor/.style={left=.6cm},
% ENLLAÇOS	
link/.style={-latex, line width=1.5pt, linkred},
link label font=\scriptsize\bfseries\color{linkred}\MakeUppercase,
link label anchor/.style={below left=-2pt and 0pt}
]{13}%
% CONTINGUT
\gantttitle[title label anchor/.style={below left=7pt and -3pt}]%
{SETMANES:\quad1}{1}
\gantttitlelist{2,...,13}{1} \\
\ganttgroup[progress=57, progress label font=\bfseries\small]%
{Objectiu 1}{1}{10} \\
\ganttbar[progress=75, name=WBS1A]%
{\textbf{Obj 1.1} Activitat A}{1}{8} \\
\ganttbar[progress=67, name=WBS1B]%
{\textbf{Obj 1.2} Activitat B}{1}{3} \\
\ganttbar[progress=50, name=WBS1C]%
{\textbf{Obj 1.3} Activitat C}{4}{10} \\
\ganttbar[progress=0, name=WBS1D]%
{\textbf{Obj 1.4} Activitat D}{4}{10} \\[grid]
\ganttgroup[progress=0, progress label font=\bfseries\small]%
{Objectiu 2}{4}{10} \\
\ganttbar[progress=0]{\textbf{Obj 2.1} Activitat E}{4}{5} \\
\ganttbar[progress=0]{\textbf{Obj 2.2} Activitat F}{6}{8} \\
\ganttbar[progress=0]{\textbf{Obj 2.3} Activitat G}{9}{10}
\ganttlink[link type=s-s]{WBS1A}{WBS1B}
\ganttlink[link type=f-s]{WBS1B}{WBS1C}
\ganttlink[link type=f-f, link label anchor/.style={left}]{WBS1C}{WBS1D}
\end{ganttchart}






\end{document}

